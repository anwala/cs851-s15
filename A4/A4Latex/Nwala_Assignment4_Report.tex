%%%%%%%%%%%%%%%%%%%%%%%%%%%%%%%%%%%%%%%%%
% Programming/Coding Assignment
% LaTeX Template
%
% This template has been downloaded from:
% http://www.latextemplates.com
%
% Original author:
% Ted Pavlic (http://www.tedpavlic.com)
%
% Note:
% The \lipsum[#] commands throughout this template generate dummy text
% to fill the template out. These commands should all be removed when 
% writing assignment content.
%
% This template uses a Perl script as an example snippet of code, most other
% languages are also usable. Configure them in the "CODE INCLUSION 
% CONFIGURATION" section.
%
%%%%%%%%%%%%%%%%%%%%%%%%%%%%%%%%%%%%%%%%%

%----------------------------------------------------------------------------------------
%	PACKAGES AND OTHER DOCUMENT CONFIGURATIONS
%----------------------------------------------------------------------------------------

\documentclass{article}

\usepackage{fancyhdr} % Required for custom headers
\usepackage{lastpage} % Required to determine the last page for the footer
\usepackage{extramarks} % Required for headers and footers
\usepackage[usenames,dvipsnames]{color} % Required for custom colors
\usepackage{graphicx} % Required to insert images
\usepackage{listings} % Required for insertion of code
\usepackage{courier} % Required for the courier font
\usepackage{lipsum} % Used for inserting dummy 'Lorem ipsum' text into the template
\usepackage{setspace}
\usepackage{color}
\usepackage{comment}
\usepackage{caption}

\usepackage{hyperref}
\usepackage{natbib}
\usepackage{underscore}

\hypersetup{
    colorlinks=true,
    linkcolor=blue,
    filecolor=magenta,      
    urlcolor=cyan,
    breaklinks=true
}

%\usepackage[]{algorithm2e}
\usepackage{pdfpages}




%For python inclusion (http://widerin.org/blog/syntax-highlighting-for-python-scripts-in-latex-documents)
\definecolor{Code}{rgb}{0,0,0}
\definecolor{Decorators}{rgb}{0.5,0.5,0.5}
\definecolor{Numbers}{rgb}{0.5,0,0}
\definecolor{MatchingBrackets}{rgb}{0.25,0.5,0.5}
\definecolor{Keywords}{rgb}{0,0,1}
\definecolor{self}{rgb}{0,0,0}
\definecolor{Strings}{rgb}{0,0.63,0}
\definecolor{Comments}{rgb}{0,0.63,1}
\definecolor{Backquotes}{rgb}{0,0,0}
\definecolor{Classname}{rgb}{0,0,0}
\definecolor{FunctionName}{rgb}{0,0,0}
\definecolor{Operators}{rgb}{0,0,0}
\definecolor{Background}{rgb}{0.98,0.98,0.98}

% Margins
\topmargin=-0.45in
\evensidemargin=0in
\oddsidemargin=0in
\textwidth=6.5in
\textheight=9.0in
\headsep=0.25in

\linespread{1.1} % Line spacing

% Set up the header and footer
\pagestyle{fancy}
\lhead{\hmwkAuthorName} % Top left header
%\chead{\hmwkClass\ (\hmwkClassInstructor\ \hmwkClassTime): \hmwkTitle} % Top center head
\chead{\hmwkClass\ (\hmwkClassInstructor): \hmwkTitle} % Top center head
\rhead{\firstxmark} % Top right header
\lfoot{\lastxmark} % Bottom left footer
\cfoot{} % Bottom center footer
\rfoot{Page\ \thepage\ of\ \protect\pageref{LastPage}} % Bottom right footer
\renewcommand\headrulewidth{0.4pt} % Size of the header rule
\renewcommand\footrulewidth{0.4pt} % Size of the footer rule

\setlength\parindent{0pt} % Removes all indentation from paragraphs

%----------------------------------------------------------------------------------------
%	CODE INCLUSION CONFIGURATION
%----------------------------------------------------------------------------------------

\definecolor{MyDarkGreen}{rgb}{0.0,0.4,0.0} % This is the color used for comments
\lstloadlanguages{Perl} % Load Perl syntax for listings, for a list of other languages supported see: ftp://ftp.tex.ac.uk/tex-archive/macros/latex/contrib/listings/listings.pdf
\lstset{language=Perl, % Use Perl in this example
        frame=single, % Single frame around code
        basicstyle=\small\ttfamily, % Use small true type font
        keywordstyle=[1]\color{Blue}\bf, % Perl functions bold and blue
        keywordstyle=[2]\color{Purple}, % Perl function arguments purple
        keywordstyle=[3]\color{Blue}\underbar, % Custom functions underlined and blue
        identifierstyle=, % Nothing special about identifiers                                         
        commentstyle=\usefont{T1}{pcr}{m}{sl}\color{MyDarkGreen}\small, % Comments small dark green courier font
        stringstyle=\color{Purple}, % Strings are purple
        showstringspaces=false, % Don't put marks in string spaces
        tabsize=5, % 5 spaces per tab
        %
        % Put standard Perl functions not included in the default language here
        morekeywords={rand},
        %
        % Put Perl function parameters here
        morekeywords=[2]{on, off, interp},
        %
        % Put user defined functions here
        morekeywords=[3]{test},
       	%
        morecomment=[l][\color{Blue}]{...}, % Line continuation (...) like blue comment
        numbers=left, % Line numbers on left
        firstnumber=1, % Line numbers start with line 1
        numberstyle=\tiny\color{Blue}, % Line numbers are blue and small
        stepnumber=5 % Line numbers go in steps of 5
}

% Creates a new command to include a perl script, the first parameter is the filename of the script (without .pl), the second parameter is the caption
\newcommand{\perlscript}[2]{
\begin{itemize}
\item[]\lstinputlisting[caption=#2,label=#1]{#1.pl}
\end{itemize}
}


%----------------------------------------------------------------------------------------
%	DOCUMENT STRUCTURE COMMANDS
%	Skip this unless you know what you're doing
%----------------------------------------------------------------------------------------

% Header and footer for when a page split occurs within a problem environment
\newcommand{\enterProblemHeader}[1]{
\nobreak\extramarks{#1}{#1 continued on next page\ldots}\nobreak
\nobreak\extramarks{#1 (continued)}{#1 continued on next page\ldots}\nobreak
}

% Header and footer for when a page split occurs between problem environments
\newcommand{\exitProblemHeader}[1]{
\nobreak\extramarks{#1 (continued)}{#1 continued on next page\ldots}\nobreak
\nobreak\extramarks{#1}{}\nobreak
}

\setcounter{secnumdepth}{0} % Removes default section numbers
\newcounter{homeworkProblemCounter} % Creates a counter to keep track of the number of problems

\newcommand{\homeworkProblemName}{}
\newenvironment{homeworkProblem}[1][Problem \arabic{homeworkProblemCounter}]{ % Makes a new environment called homeworkProblem which takes 1 argument (custom name) but the default is "Problem #"
\stepcounter{homeworkProblemCounter} % Increase counter for number of problems
\renewcommand{\homeworkProblemName}{#1} % Assign \homeworkProblemName the name of the problem
\section{\homeworkProblemName} % Make a section in the document with the custom problem count
\enterProblemHeader{\homeworkProblemName} % Header and footer within the environment
}{
\exitProblemHeader{\homeworkProblemName} % Header and footer after the environment
}

\newcommand{\problemAnswer}[1]{ % Defines the problem answer command with the content as the only argument
\noindent\framebox[\columnwidth][c]{\begin{minipage}{0.98\columnwidth}#1\end{minipage}} % Makes the box around the problem answer and puts the content inside
}

\newcommand{\homeworkSectionName}{}
\newenvironment{homeworkSection}[1]{ % New environment for sections within homework problems, takes 1 argument - the name of the section
\renewcommand{\homeworkSectionName}{#1} % Assign \homeworkSectionName to the name of the section from the environment argument
\subsection{\homeworkSectionName} % Make a subsection with the custom name of the subsection
\enterProblemHeader{\homeworkProblemName\ [\homeworkSectionName]} % Header and footer within the environment
}{
\enterProblemHeader{\homeworkProblemName} % Header and footer after the environment
}

%----------------------------------------------------------------------------------------
%	NAME AND CLASS SECTION
%----------------------------------------------------------------------------------------

\newcommand{\hmwkTitle}{Assignment\ \#4 } % Assignment title
%\newcommand{\hmwkDueDate}{Monday,\ January\ 1,\ 2012} % Due date
\newcommand{\hmwkClass}{Introduction to Web Science} % Course/class
%\newcommand{\hmwkClassTime}{10:30am} % Class/lecture time
\newcommand{\hmwkClassInstructor}{Dr. Nelson} % Teacher/lecturer
\newcommand{\hmwkAuthorName}{Alexander Nwala} % Your name

%----------------------------------------------------------------------------------------
%	TITLE PAGE
%----------------------------------------------------------------------------------------

\title{
\vspace{2in}
\textmd{\textbf{\hmwkClass:\ \hmwkTitle}}\\
%\normalsize\vspace{0.1in}\small{Due\ on\ \hmwkDueDate}\\
%\vspace{0.1in}\large{\textit{\hmwkClassInstructor\ \hmwkClassTime}}
\vspace{0.1in}\large{\textit{\hmwkClassInstructor}}
\vspace{3in}
}

\author{\textbf{\hmwkAuthorName}}
\date{Friday, May 1, 2015} % Insert date here if you want it to appear below your name

%----------------------------------------------------------------------------------------

\begin{document}

\maketitle



%----------------------------------------------------------------------------------------
%	TABLE OF CONTENTS
%----------------------------------------------------------------------------------------

%\setcounter{tocdepth}{1} % Uncomment this line if you don't want subsections listed in the ToC

\newpage
\tableofcontents
\newpage

%----------------------------------------------------------------------------------------
%	PROBLEM 1
%----------------------------------------------------------------------------------------

% To have just one problem per page, simply put a \clearpage after each problem

\begin{homeworkProblem}
Using the pages from A3 that boilerpipe successfully processed, download those representations again \& reprocess them with boilerpipe.\\
Document the time difference (e.g., Time(A4) – Time(A3)).
Compute the Jaccard Distance x for each pair of pages 
(i.e., P(A3) \& P(A4) for:
\begin{enumerate}
    \item Unique terms (i.e., unigrams)
    \item Bigrams
    \item Trigrams
\end{enumerate}
See: \url{http://en.wikipedia.org/wiki/Jaccard_index}.
For each of the 3 cases (i.e., 1-, 2-, 3-grams) build a Cumulative Distribution Function that shows the \% change on the x-axis \& the \% of the population on the x-axis.
See: \url{http://en.wikipedia.org/wiki/Cumulative_distribution_function}.
Give 3-4 examples illustrating the range of change that you have measured.\\

%\lstinputlisting[breaklines=true, caption=Curl Demo]{"/home/anwala/CS 895/Assignment 1/problem1_curlDemonstration.py"}
%\lstinputlisting[breaklines=true, caption=Hash function; extract HTML funtion; and strip HTML tags function]{hashExtractProcessHTMlSnippet.py}


\begin{comment}
\begin{figure}
    \caption{curlDemoOutput}
    \begin{center}
        \includegraphics{curlDemo} % Example image
    \end{center}
\end{figure}
\end{comment}

%\problemAnswer
%{
    


    \textbf{SOLUTION 1}\\

    The solution for this problem is outlined by the following steps:\\

    \begin{enumerate}
    \item \textbf{Download pages a second time:}
    The first set of text (\textbf{linksFile.txt}) was downloaded on March 29, 2015 and the second was downloaded on April 10, 2015 (12 days apart).\\

    \item \textbf{Tokenize and calculate n-grams:} Due to Listing 2. the text downloaded in 1. was to tokenized and 1, 2, 3-grams calculated for both sets (text downloaded on March 29, 2015 and April 10, 2015).\\

    \item \textbf{Calculated Jaccard distance:} Also Due to Listing 2. the Jaccard distance was calculated for every pair in the set across the 1, 2, 3-grams.\\

    \item \textbf{Plot Cummulative Distribution Function For Jaccard Distance:} Chart 1. was produced due to Listing 1.
    \end{enumerate}

    \lstinputlisting[breaklines=true, caption=Plot CDF for Jaccard Distance values]{../A4/cdf.r}

    \includepdf[scale=0.80]{../A4/RplotsSim.pdf}

    \lstinputlisting[breaklines=true, caption=Calculate Jaccard Distance]{jaccardDistanceSnippet.py}

    \begin{verbatim}
        The file nGramSimilarity.txt contains the similarity values
    \end{verbatim}

    

    
    
    
%}



\end{homeworkProblem}

%----------------------------------------------------------------------------------------
%	PROBLEM 2
%----------------------------------------------------------------------------------------

\begin{homeworkProblem}

Using the pages from Q1 (A4), download all TimeMaps (including TimeMaps with 404 responses, i.e. empty or null TimeMaps).
Upload all the TimeMaps to github
Build a CDF for \# of mementos for each original URI (i.e., x-axis = \# of mementos, y-axis = \% of links)
See: \url{http://timetravel.mementoweb.org/guide/api/}\\

\textbf{SOLUTION 2}\\

The solution for this problem is outlined by the following steps:\\

\begin{enumerate}

\item \textbf{Download all timemaps:} Due to Listing 4, the timemaps for all URLs was downloaded based on the memento API as implemented by \url{https://github.com/anwala/wdill}. For each URI, Listing 4. paginates through all the timemaps and dereferences each in order to count the mementos.\\

\item \textbf{Plot CDF for \# of mementos:} Listing 3. plots the CDF for all the count of mementos by using R's primitive Empirical Cummulative Distribution function (ECDF). As seen from Chart 2, the distribution tells a story or two extremes: most of the URIs have 0 mementos due to the fact that they are new, and have had insufficient chances to be archived. On the otherhand, a few of the URIs are very popular, hence had sufficient time to be archived.\\

\end{enumerate}

\begin{verbatim}
    The folder Timemaps contains the timemaps
\end{verbatim}

\lstinputlisting[breaklines=true, caption=Plot CDF for Memento Count]{../A4/cdf2.r}

\lstinputlisting[breaklines=true, caption=Count Mementos for all URIs]{countMementosSnippet.py}

\includepdf[scale=0.80]{../A4/RplotsCDF.pdf}


\begin{verbatim}
    The file mementoCount.txt contains the count of all
    mementos for all the URIs in linksFile.txt.
\end{verbatim}


\end{homeworkProblem}


%----------------------------------------------------------------------------------------
%   PROBLEM 3
%----------------------------------------------------------------------------------------

\begin{homeworkProblem}

Using 20 links that have TimeMaps
With \textgreater= 20 mementos, 
have existed \textgreater= 2 years (i.e., Memento-Datetime of ``first memento'' is April XX, 2013 or older).
Note: select from Q1/Q2 links, else choose them by hand.\\ \\
For each link, create a graph that shows Jaccard Distance, relative to the first memento, through time
x-axis: continuous time, y-axis: Jaccard Distance relative to the first memento\\

\textbf{SOLUTION 3}\\

The solution for this problem is outlined by the following steps:\\

\begin{enumerate}
\item \textbf{Download first memento:} This was achieved due to Listing 5. which downloads all the mementos for a given URI and sorts the entries in ascending order based on datetime values of the mementos.\\

Consider the following: based on the initial list of URIs, only 21 had mementos, but only 4 (10790, 25, 29, 22) had memento count exceeding 20. From this short list, boilerplate removal was only successful for two (\url{http://tinyurl.com/} and \url{http://www-01.ibm.com/software/analytics/solutions/customer-analytics/social-media-analytics/}). The solution to part 3 of this assignment addressed the two URIs. However, the solution could scale to address a larger set.


\lstinputlisting[breaklines=true, caption=Get First Memento]{getFirstMementoSnippet.py}

\item \textbf{Download text of first memento URI:} Due to Listing 6. based on justText \cite{justText} boilerpipe, the text for the first memento was downloaded.

\lstinputlisting[breaklines=true, caption=Extract Text for URI of first Memento]{extractTextSnippet.py}

\item \textbf{Tokenize and compute similarity:} Due to Listing 7. which is similar to Listing 2. the 1-gram similarity was calculate between all mementos relative to the first memento for the selected URIs.

\lstinputlisting[breaklines=true, caption=Calculate Jaccard Distance Relative to first Mementos]{calculateRelSimilaritySnippet.py}

\item \textbf{Plot Jaccard Distance:} Due to Listing 8. Chart's 3 and 4 were produced.
\end{enumerate}

\lstinputlisting[breaklines=true, caption=Plot Jaccard Distance Relative to first Mementos]{../A4/plotSim.r}

\includepdf[scale=0.80]{../A4/url2Sim.pdf}
\includepdf[scale=0.80]{../A4/url1Sim.pdf}

\begin{verbatim}
    The files sim1.txt and sim2.txt contains the similarity values
\end{verbatim}

\end{homeworkProblem}

\bibliographystyle{plain}
\bibliography{A4BibFile}

%----------------------------------------------------------------------------------------

\end{document}